% %	Este arquivo não esta oroginalmente incluso e representa apenas um
% %	breve modelo de como se produzir uma capa personalizada

% \begin{titlepage}
% \centering
% % \doublespacing
% \scshape
% \normalfont
% \vspace{0.1\textheight}
% \vbox{\normalfont{UNB - UNIVERSIDADE DE BRASILIA\\CAMPUS DARCY RIBEIRO}}
% \vspace{0.1\textheight}


% \vbox{\Huge
% %Nome do Trabalho

% \ver

% \vspace{0.03\textheight}
% \hrule }

% \vbox{
% %Nome da Matéria

% \hell

% }
% \vspace{0.3\textheight}

% %teste

% \vspace{0.3\textheight}
% %Nomes
% \vbox{\scshape
% {\names}}
% \end{titlepage}

\begin{tabular*}{\textwidth}{l @{\extracolsep{\fill}} r @{\extracolsep{6pt}} l}
\textbf{\class} &&\\ 
\textbf{\examnum} & \textbf{Nome: }\name &\\
\textbf{\examdate} &\textbf{Matricula:} \luizmatricula &
\end{tabular*}\\
\rule[2ex]{\textwidth}{2pt}
%redefinição da saida para respostas
\renewcommand{\solutiontitle}{\noindent\textbf{Resposta:}\par\noindent}
\printanswers %apresentar respostas
%\addpoints % points count
\noaddpoints % to omit double points count