% \documentclass[journal,compsoc]{IEEEtran}\newcommand{\journal}{true}
\documentclass[conference]{IEEEtran}\newcommand{\journal}{false}
% \documentclass[12pt,openright,twoside,a4paper,english,french,spanish]{abntex2}
% \documentclass[a4paper,14pt]{report}
% \documentclass[a4paper]{coursepaper}
% \documentclass[12pt,answers]{exam}
% \documentclass[a4paper,11pt]{article}
% \documentclass[a4paper,11pt]{book}
% \documentclass[12pt,addpoints]{exam}
\input packages

\input nomes
\input setup
\hyphenation{MAT-LAB}

\begin{document}
% \input capa
\onecolumn
\maketitle
%status atual do seu projeto, descrevendo o que já foi implementado, o que está faltando e as dificuldades que você está encontrando. 

\section*{Status} % (fold)
\label{sec:status}
    

    \subsection*{O que já foi implementado} % (fold)
    \label{sub:o_que_já_foi_implementado}

    \begin{itemize}
        \item Escolhido tema.
        \item Estudo de bibliografia.
        \item Seleção de ferramentas e bibliotecas para auxilio.
        \item Banco de imagens para testes (\url{http://deeplearning.net/datasets/}).
        \item Centralização do trabalho em um repositório.
    \end{itemize}
    
    \subsection*{O que ainda falta} % (fold)
    \label{sub:o_que_ainda_falta}

    \begin{itemize}
        \item Desenvolvimento do software.
        \item Análise dos resultados.
    \end{itemize}
    
    \subsection*{Dificuldades} % (fold)
    \label{sub:dificuldades}
    

    \begin{itemize}
        \item Tempo.
        \item Domínio das ferramentas.
    \end{itemize}

\section*{Acompanhamento do desenvolvimento} % (fold)
\label{sec:acompanhamento_do_desenvolvimento}

    O acompanhamento do desenvolvimento do trabalho pode ser observado no repositório onde se encontrará o projeto final, em \url{https://github.com/Ziul/pdi-final}. A política de \textit{branchs} esta sendo utilizada para auxiliar no desenvolvimento do trabalho e concentração dos artefatos para o trabalho.

\end{document}
