
\section*{descrição do problema}

Este trabalho tem como objetivo a criação de um software de reconhecimento facial e sua descrição. Devido ao tempo e recursos disponíveis para a produção deste trabalho, será aceito uma baixa taxa de qualidade no reconhecimento. Espera-se como resultados que o software seja estável, consiga identificar rostos em uma imagem e possa destaca-los para a apresentação de resultados.

\section*{estado da arte}

% http://www.linhadecodigo.com.br/artigo/1813/biometria-reconhecimento-facial-livre.aspx
Um dos processos de identificação mais utilizados pelos seres humanos é o reconhecimento facial, o qual permite identificar rapidamente qualquer pessoa e assim definir o tipo apropriado de interação com ela. O reconhecimento facial ainda nos oferece a possibilidade de perceber o estado emocional de uma pessoa, contribuindo para o relacionamento.

O reconhecimento facial é extremamente complexo de implementar em uma máquina, visto não sabermos ao certo como o cérebro humano realiza essa tarefa. O cérebro humano pode identificar corretamente uma pessoa a partir de sua imagem facial mesmo sobre as mais diversas condições, como variações de iluminação, observando apenas uma de suas características ou partes, e até mesmo com distorções ou deformações.

\section*{abordagem escolhida}

\lipsum[3]

\section*{resultados e simulações}

\lipsum[4]

\section*{conclusões e trabalhos futuros}

\lipsum[5]