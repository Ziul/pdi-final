Este trabalho foi implementado majoritariamente com linguagem \textbf{Python} por oferecer facilidade em prototipagem de software e uma vasta gama de bibliotecas para auxilio de estruturas de dados, porém um \textit{bind} com MATLAB esta sendo oferecido.

\section*{Codificação} % (fold)
\label{sub:codificação}

A codificação do dado ocorre, tendo como base os símbolos e o intervalo de cada um deles. Este intervalo é definido com base na probabilidade de cada símbolo. Tendo essa probabilidade já definida, monta-se uma tabela com o seguinte formato:

\begin{center}
{\ttfamily
    MAP[símbolo] = [probabilidade]\\
    MAP[símbolo] = [intervalo\_inferior, intervalo\_superior]
}
\end{center}

A sequencia de símbolos a ser codificada, junto com a tabela de intervalos é então passada para o método {\ttfamily encode}, o qual retorna o número codificado. Atualmente esse resultado esta limitado à 64 bits.
% section codificação (end)

\section*{Decodificação} % (fold)
\label{sub:decodificação}
Análogo a codificação, a decodificação necessita do código a ser decodificado e da tabela com o intervalo de valores. Novamente, essa tabela pode ser construída apenas com a probabilidade de cada símbolo. Com ambas as informações, basta fazer a chamada do método {\ttfamily decode} passando como parâmetro o código e a tabela. Como resposta, o método retorna a sequencia de símbolos a qual o código representa com aquela tabela de probabilidade.
% section decodificação (end)

\section*{Performance} % (fold)
\label{sub:performance}

% \begin{table}[htbp]
% \centering
% \caption{}
% \begin{longtable}{|c|p{2cm}|c|c|}
% \hline
% \textbf{Entrada} & \textbf{tipo} & \textbf{tamanho} & \textbf{tempo[s]} \\ \hline
% “main.tex” & texto & 8 bytes & 1.377 \\ \hline
% {\ttfamily README.md} & arquivo de texto & 269 bytes & 1.392 \\ \hline
% {\ttfamily resumo.pdf} & texto + binário & 116 bytes & 1.402 \\ \hline
% {\ttfamily fga.jpg} & imagem & 45 Kbytes & 1.421 \\ \hline
% {\ttfamily file.log} & texto & 852 bytes & 1.442 \\ \hline
% {\ttfamily lena.jpg} & imagem & 1.2 Mb & 2.179 \\ \hline
% {\ttfamily livro.pdf} & texto + binário & 18 Mb & 13.065 \\ \hline
% \end{longtable}
\label{}
% \end{table}

% section performance (end)

\section*{Bind de Python com MATLAB} % (fold)
\label{sec:bind_com_matlab}

O bind de Python com MATLAB esta disponível para as versões do MATLAB R2014B\footnote{Informações segundo \url{http://www.mathworks.com/help/matlab/matlab_external/system-and-configuration-requirements.html}.} em diante. Um exemplo do uso do \textit{bind} pode ser observado no \autoref{encoding}.

\begin{lstlisting}[language=MATLAB,numbers=none,basicstyle=\footnotesize,label=encoding,caption={Chamada de funções python no MATLAB}]
% clear classes, scripts, etc
clear classes;

% Import module
mod = py.importlib.import_module('arithmetic');

% Some source and code example
source_code = 'arithmetic';
code = 0.0757451536000

% Setup the probabilities and the rage for each symbol
probabilities = py.arithmetic.set_probability(source_code);
symbols_with_range = py.arithmetic.set_range(probabilities);

% Encoding
result = py.arithmetic.encode(source_code, probabilities);

% Decoding
result = py.arithmetic.encode(code, probabilities);
\end{lstlisting}

% section bind_com_matlab (end)

\section*{Problemas e dificuldades} % (fold)
\label{sec:problemas_e_dificuldades}

Um dos maiores problemas é o armazenamento e leitura do número codificado. Devido a arquitetura ter uma limitação de 64 bit, as linguagens tem por natureza a escrita limitada a este tamanho também, sendo necessário a escrita em bits diretamente. Todavia este é um número flutuante, e não um inteiro, fazendo seu conteúdo mais complexo de ser gerenciado.

Outro grande problema é o controle de operações com ponto flutuante, visto que perdas e arredondamento são comuns quando o número é menor que o ponto de arredondamento. Para contornar este problema, uma biblioteca externa dedicada a evitar estes problema esta sendo utilizada.

% section problemas_e_dificuldades (end)

\section*{Disponibilidade} % (fold)
\label{sec:disponibilidade}
O código na integra pode ser obtido em \url{https://github.com/Ziul/arithmetic_coding}. A \textit{branch} {\ttfamily master} contém a biblioteca em Python, enquanto a \textit{branch} {\ttfamily Matlab} contem um exemplo de uso com o MATLAB.
% section disponibilidade (end)