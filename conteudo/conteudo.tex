
\section*{descrição do problema}

Este trabalho tem como objetivo a criação de um software de reconhecimento facial e sua descrição. Devido ao tempo e recursos disponíveis para a produção deste trabalho, será aceito uma baixa taxa de qualidade no reconhecimento. Espera-se como resultados que o software seja estável, consiga identificar rostos em uma imagem e possa destaca-los para a apresentação de resultados.

\section*{estado da arte}

% http://www.linhadecodigo.com.br/artigo/1813/biometria-reconhecimento-facial-livre.aspx
Um dos processos de identificação mais utilizados pelos seres humanos é o reconhecimento facial, o qual permite identificar rapidamente qualquer pessoa e assim definir o tipo apropriado de interação com ela. O reconhecimento facial ainda nos oferece a possibilidade de perceber o estado emocional de uma pessoa, contribuindo para o relacionamento.

O reconhecimento facial é extremamente complexo de implementar em uma máquina, visto não sabermos ao certo como o cérebro humano realiza essa tarefa. O cérebro humano pode identificar corretamente uma pessoa a partir de sua imagem facial mesmo sobre as mais diversas condições, como variações de iluminação, observando apenas uma de suas características ou partes, e até mesmo com distorções ou deformações.

% Face recognition based on the geometric features of a face is probably the most intuitive approach to face recognition. One of the first automated face recognition systems was described in [Kanade73]: marker points (position of eyes, ears, nose, ...) were used to build a feature vector (distance between the points, angle between them, ...). The recognition was performed by calculating the euclidean distance between feature vectors of a probe and reference image. Such a method is robust against changes in illumination by its nature, but has a huge drawback: the accurate registration of the marker points is complicated, even with state of the art algorithms. Some of the latest work on geometric face recognition was carried out in [Bru92]. A 22-dimensional feature vector was used and experiments on large datasets have shown, that geometrical features alone my not carry enough information for face recognition.

% The Eigenfaces method described in [TP91] took a holistic approach to face recognition: A facial image is a point from a high-dimensional image space and a lower-dimensional representation is found, where classification becomes easy. The lower-dimensional subspace is found with Principal Component Analysis, which identifies the axes with maximum variance. While this kind of transformation is optimal from a reconstruction standpoint, it doesn’t take any class labels into account. Imagine a situation where the variance is generated from external sources, let it be light. The axes with maximum variance do not necessarily contain any discriminative information at all, hence a classification becomes impossible. So a class-specific projection with a Linear Discriminant Analysis was applied to face recognition in [BHK97]. The basic idea is to minimize the variance within a class, while maximizing the variance between the classes at the same time.

% Recently various methods for a local feature extraction emerged. To avoid the high-dimensionality of the input data only local regions of an image are described, the extracted features are (hopefully) more robust against partial occlusion, illumation and small sample size. Algorithms used for a local feature extraction are Gabor Wavelets ([Wiskott97]), Discrete Cosinus Transform ([Messer06]) and Local Binary Patterns ([AHP04]). It’s still an open research question what’s the best way to preserve spatial information when applying a local feature extraction, because spatial information is potentially useful information.

\section*{abordagem escolhida}

% http://docs.opencv.org/2.4/modules/contrib/doc/facerec/facerec_tutorial.html
\lipsum[3]

\section*{resultados e simulações}

\lipsum[4]

\section*{conclusões e trabalhos futuros}

\lipsum[5]